\documentclass[11pt]{article}
\usepackage{listings}
\usepackage{hyperref}
\usepackage[utf8]{inputenc}
\usepackage{fancyhdr}
\usepackage[margin=0.9in]{geometry}
\usepackage{color, colortbl}

\definecolor{hgray}{gray}{0.9}
 
\pagestyle{fancy}
\def\sectionautorefname{Section}


\title{nomodo - Manage your system by yourself}
\author{Autori (in ordine alfabetico): \and Giuseppe Glorioso \and Lucia Polizzi}

\begin{document}
\lstset{
	language=Python,
	frame=single,
	breaklines=true,
	breakatwhitespace=false,
	backgroundcolor=\color{hgray}
}
\maketitle
\pagenumbering{gobble}

\newpage
\tableofcontents
\pagenumbering{arabic}

\newpage

\section{Introduzione}

\subsection{Introduzione al progetto}
Il progetto nomodo nasce dalla necessit\'a di un applicativo di gestione dei sistemi Ubuntu che sia più immediato ed accessibile
rispetto al classico terminale, e quindi utilizzabile anche dagli utenti che per un motivo o per un altro
non possono o non vogliono avere a che fare con il terminale.
Nomodo si prende in carico di eseguire tutte le chiamate al terminale o meno per eseguire operazioni atte alla gestione del sistema
presentando all'utente una interfaccia web chiara e comprensibile.
Per operazioni in questo caso si intendono l'aggiornamento, la manutenzione e il miglioramento del sistema come ad esempio
l'installazione dei pacchetti, la ricerca e la modifica dei file, così come operazioni di più alto livello come
la gestione basilare del web server Apache.

\subsection{Informazioni tecniche}
\paragraph{Python + Flask}
L'applicativo scritto in python è basato sul framework Flask, utilizzato tra l'altro come webserver per l'accesso al pannello.
Durante la fase di sviluppo si è utilizzato nginx come reverse proxy in modo da poter raggiungere il pannello web sulla porta 80 e non sulla 5000.
È stata presa poi in seguito la decisione di lasciare che l'applicativo girasse sulla porta 5000 in quannto meno comune e quindi
meno alla mercé degli hacker.
\\~\\
L'applicazione è stata quindi divisa in modo netto nelle due componenti fondamentali, il \nameref{frontend} e il \nameref{backend}
che anche andremo quindi ad analizzare qui brevemente e più approfinditamente nei capitoli successivi:
\begin{itemize}
	\label{backendintro}
	\item{
			Il \textbf{backend} consiste in una serie di funzioni raccolte in una serie di file a mò di libreria,
			risiedenti nella cartella \texttt{systemcalls} (come ad es. \texttt{system.py} o \texttt{user.py}), utilizzati
			sia per la raccolta di dati sia per eseguire azioni sul sistema che non necessitano di output in uscita}
	\label{frontendintro}
	\item{
			Il \textbf{frontend} rappresenta la parte grafica dell'applicativo web, e utilizza le funzioni
			del backend per la ricerca di informazioni e per la modifica alle componenti del sistema
			inclusa la modifica dei file quali i file di configurazioni}
\end{itemize}

\paragraph{MongoDB}
Ogni operazione sensibile effettuata tramite l'applicazione comporta la memorizzazione delle modifiche che comporta la stessa
in documento di mongodb, cos\'i da poter risalire alla storia delle operazioni effettuate e tentare un revert delle modifiche
in caso ad esempio il sistema perda di stabilità o le modifiche non portino al risultato sperato.
Tali operazioni possono riguardare ad esempio la modifica di un file, o la rimozione di un pacchetto dal sistema.
Ogni log in mongodb presenta inoltre un flag \textbf{status} che indica se l'operazione eseguita sia andata a buon fine
o meno, in modo da rendere più chiara la navigazione tra i log e dare la possibiltà all'utente di filtrarli in base a questo campo.
\footnote{\label{loglimit1}
Le uniche operazioni memorizzate nel database sono quelle relative all'utilizzo dell'applicativo;
una modifica effettuata direttamente sul sistema ad esempio tramite il terminale va incontro alle regole del sistema Ubuntu
e ogni modifica potrebbe essere irreversibile. In questi casi fare riferimento ai log del sistema che è possibile trovare
al percorso \texttt{/var/log/} o sul pannello web alla sezione \texttt{Log}. }

\section{Backend}\label{backend}
Come anticipato in sezione \ref{backendintro} il backend è composto da una serie di funzioni raggruppate per categoria
che fanno utilizzo di varie librerie python per compiere operazioni che possono o meno alterare lo stato del sistema.
Allo stato attuale le categorie che compongono il backend sono le seguenti:
\begin{itemize}
	\item{Utenti}
	\item{Network}
	\item{Cron}
	\item{Sistema}
	\item{Apache}
	\item{Database}
	\item{File}
	\item{Logs}
\end{itemize}

\paragraph{Interfaccia al frontend}\label{frontendinterface}
Ogni funzione chimata restituisce sempre un dizionario contenente almeno n codice di ritorno e il logid del documento inserito
in mongo con un mongo \texttt{\_id} se applicabile\footnote{\label{mongologwhen}
Cioè in caso l'operazione sia una operazione sensibile e richieda quindi un inserimento in mongo per tenere traccia della stessa}
oppure un logid \texttt{None} se non è stato creato alcun log.
Distinguiamo quindi 2 casi in base al valore della variabile \texttt{returncode}:
\begin{itemize}
	\item{Se $returncode = 0$ l'operazione è andata a buon fine e il dizionario conterrà una terza variabile \texttt{data}
		che conterrà i dati richiesti se la funzione chiamata è tesa per restutuire output oppure sarà una variabile nulla
		se la funzione non restituisce output}
	\item{Se $returncode \neq 0$ c'è stato un errore durante l'esecuzione dell'applicazione e il dizionario restituito conterrà
		quindi una terza variabile \texttt{stderr} il cui valore è un messaggio di errore e, se l'errore è dato da un comando
		eseguito in bash, il comando che una volta lanciato ha generato l'eccezione.}
\end{itemize}
Il frontend o l'utente che voglia chiamare per qualsivoglia motivo le funzioni del backend direttamente, potr\'a farlo quindi nel seguente modo::
\begin{lstlisting}
data = getifacestat()
if data['returncode'] is 0:
	data = data['data']
else:
	print( data['stderr'] )
 
pprint(data)
\end{lstlisting}

\paragraph{subprocess}\label{subprocess}
Le funzionalità di Python più utilizzata per la realizzazione dell'applicazione sono senza dubbio quelle appartenenti alla libreria subprocess,
che permette di eseguire comandi come se si stessero eseguendo in bash. Si è cercato il più possibile di limitare l'utilizzo di questa libreria
ma le sue funzionalità si sono rese necessarie nella maggior parte dei casi delle funzioni del backend,
a causa dello scarsa agilità che ha python di interfacciarsi col sistema sottostante.
In generale l'esecuzione di un comando avviene nel seguente modo:
\begin{lstlisting}
command = ['ifconfig', '-a']
try:
	output = check_output(command, stderr=PIPE, universal_newlines=True)
except CalledProcessError as e:
	return command_error(e, command, logid)

return command_success( output )
\end{lstlisting}
Tutte le funzioni di nomodo ritornano o con un \texttt{ \nameref{command\string_success}} in caso l'operazione sia andata a buon fine
o con un \texttt{ \nameref{commmand\string_error} } in caso il comando non vada a buon fine e venga lanciata l'eccezione \texttt{CalledProcessError}.
In entrambi i casi viene restituito il dizionario menzionato in sezione \ref{frontendinterface}.
Un esempio di comando che non restituisce output è il seguente:
\begin{lstlisting}
def removeuser(user, removehome=None):
        
    logid = mongolog( locals(), getuser(user) )
    

    try:
        command = ['deluser', user]
        if removehome: command.append('--remove-home') 

        check_output( command, stderr=PIPE, universal_newlines=True )
    except CalledProcessError as e:
        return command_error(e, command, logid)
    
    
    return command_success(logid)
\end{lstlisting}
Nelle prossime sezioni verranno analizzate tutte le categorie e spiegato il funzionamento di ogni funzione che contengono.

\subsection{utilities}
Questa categoria contiene per la maggior parte funzioni che non vengono mai richiamate direttamente dal frontend, ma vengono utilizzate
dalle altre funzioni del backend. Fa eccezione la funzione \nameref{filedit} utile alla modifica di file. \\
Analizziamo le funzioni di questa libreria.
\subsubsection{mongolog()}\label{mongolog}
\begin{lstlisting}
def mongolog(params, *args):

    dblog = dict({
    	'date': datetime.datetime.utcnow(),     #Operation date
    	'funname': inspect.stack()[1][3],       #Function name
    	'parameters': params,			#Called function's parameters
    })
    
    for arg in args:
        dblog.update( arg )

    #ObjectID in mongodb
    return db.log.insert_one( dblog ).inserted_id
\end{lstlisting}
Viene chiamata ogni volta che una funzione sia classificata come \textbf{sensibile} cioè che va a modificare lievemente o pesantemente il sistema
e prende in carico di creare un log mongodb contenente le operazioni eseguite e i dati modificati dalla funzione.
Accetta N parametri di cui il primo (obbligatorio) è la lista di parametri con cui è stata lanciata la funzione
di cui si sta memorizzando il log. Ad esempio in
\begin{lstlisting}
def ifacedown( iface ):
	logid = mongolog( locals() )
	...
\end{lstlisting}
il primo parametro è \texttt{locals()} che contiene la variabile \texttt{iface} che verrà quindi memorizzata nel log di mongo;
il secondo parametro (opzionale) può essere uno o più dizionari da unire al dizionario memorizzato in mongodb.
Ad esempio nella funzionei \texttt{addusertogroups}:
\begin{lstlisting}
def addusertogroups(user, *groups):

    #Logging operation to mongo first
    userinfo = getuser(user)
    if userinfo['returncode'] is 0:
        userinfo = userinfo['data']
    else:
        return userinfo

    logid = mongolog( locals(), userinfo )
    ...
\end{lstlisting}
Si è deciso che prima di aggiungere un utente a dei nuovi gruppi si va a memorizzare in mongo non solo \texttt{locals()} e quindi
\texttt{user} e \texttt{*groups} ma anche le informazioni sull'utente ricavate attraverso la funzione \texttt{getuser()} e passate
a \texttt{mongolog()} come secondo parametro.
\\~\\
Il dizionario di base memorizzato in mongo è formato da tre elementi:
\begin{itemize}
	\item{La data in cui viene effettuata l'operazione}
	\item{Il nome della funzione che ha chiamato \texttt{mongolog}, ricavata tramite il supporto della libreria \texttt{inspect}}
	\item{I parametri della funzione che chiama, come spiegato in precedenza, e ottenuti chiamando la funzione \texttt{locals()}}
\end{itemize}

\subsubsection{mongologstatus() e funzioni collegate }\label{mongologstatus}
\begin{lstlisting}
def mongologstatus(logid, status):

    return db.log.update_one(
        {'_id': logid},
        {'$set': { 'status' : status }},
        upsert=False
        )
	
def mongologstatuserr(logid, status='error'):
    return mongologstatus(logid, status)
def mongologstatussuc(logid, status='success'):
    return mongologstatus(logid, status)
\end{lstlisting}
\paragraph{Parametri}
Accetta 2 parametri:
\begin{itemize}
	\item{\texttt{logid}: è il logid del documento di mongo a cui aggiungere o modificare il campo status}
	\item{\texttt{status='error'}: è lo stato da assegnare la log individuato da \texttt{logid}}
\end{itemize}
Questa funzione è intesa per aggiungere o modificare il campo \texttt{status} di un log di MongoDB. Le funzioni di nomodo (come è giusto che sia)
creano un documento di mongo per memorizzare le informazioni sull'operazione prima di procedere all'operazione stessa.
In caso un'operazione non andasse nel modo aspettato bisognerebbe quindi marcare il documento appena creato in mongo in modo da avvisare l'utente
che sta consultando il log che l'operazione riferita a quel documento non è andata a buon fine.
La memorizzazione del log avviene quindi nei seguenti step:
\begin{enumerate}
	\item{Viene lanciata la funzione che richiede la memorizzazione del log e quindi \nameref{mongolog}, che va a creare il log
		senza nessuna indicazione sul successo o meno dell'operazione}
	\item{Dopo l'esecuzione della funzione viene chiamata \texttt{command\string_success} se l'operazione è andata a buon fine;
		la prima operazione che questa va ad eseguire è chiamare a sua volta la funzione \texttt{mongologstatussuc()}
		che chiama \texttt{mongologstatus()} con il parametro \texttt{status='error'} aggiungendo tale campo \texttt{status} al log
		di mongo ed indicando la buona riuscita dell'applicazione all'utente che andrà ad analizzare i log}
	\item{In caso invece la funzione vada in errore viene chiamata \nameref{command\string_success} che chiama \texttt{mongologstatuserr()}
		che chiama \texttt{mongologstatus()} con il secondo parametro \texttt{status='error'} aggiungendo tale campo al log di mongo}
	\item{In caso invece si voglia personalizzare il campo \texttt{status} basta quindi che la funzione chiami direttamente \texttt{mongologstatus()}
		con il secondo parametro \texttt{status} ad un qualsivoglia valore si voglia inserire, ad es. \texttt{status='canceled'}}
\end{enumerate}
Si intuisce quindi da questi step che un log che non abbia il campo status indica un crash della funzione nel codice che è intercorso tra
la memorizzazione del log e l'aggiunta del campo \texttt{status}.
\\~\\
L'operazione deve fallire se il documento indicato da \texttt{logid} non esiste, quindi si è aggiunta la direttiva \texttt{upsert=False}.
\\~\\
Ecco un esempio che mostra lo stato di un log appena aggiunto (senza il campo \texttt{status}) e al termine dopo aver chiamato \texttt{command\string_success}:
\begin{lstlisting}
> db.log.find()
{ "_id" : ObjectId("5ae596d4bf3bd205c1aeea25"), "parameters" : { "shell" : "/bin/bash", "user" : "giuseppe2", "password" : "test" }, "funname" : "adduser", "date" : ISODate("2018-04-29T09:56:36.627Z") }
> db.log.find()
{ "_id" : ObjectId("5ae596d4bf3bd205c1aeea25"), "parameters" : { "shell" : "/bin/bash", "user" : "giuseppe2", "password" : "test" }, "funname" : "adduser", "date" : ISODate("2018-04-29T09:56:36.627Z"), "status" : "success" }
\end{lstlisting}
\paragraph{Return}
Restituisce un oggetto della classe 


\subsubsection{command\_success}\label{command\string_success}
\begin{lstlisting}
def command_success( data=None, logid=None, returncode=0 ):

    if logid:
        mongologstatussuc( logid )

    return dict({
        'returncode': returncode,
        'data': data,
        'logid': logid
    })
\end{lstlisting}
La funzione \texttt{command\_success} fondamentalmente costruisce il dizionario da restituire all'utente quando una funzione del backend
ha finito le sue opoerazioni e non ci sono stati errori durante l'esecuzione.
Insieme alla sorella \nameref{command\string_error} sono le uniche due funzioni chiamate al termine di una funzione del backend.
\paragraph{Parametri}
Accetta tre parametri:
\begin{itemize}
	\item{\texttt{data=None}: Sono i dati da restituire all'utente se la funzione che l'ha chiamata li genera. Di base è \texttt{None}}
	\item{\texttt{logid}: Il logid a cui aggiungere il campo \texttt{status} e da restituire all'utente nel dizionario come campo del dizionario.
		Di base è \texttt{None} in quanto il chiamante potrebbe non aver generato un mongolog per l'operazione che ha effettuato}
	\item{\texttt{returncode}: È il codice di ritorno che verrà inserito nel dizionario.
		Essendo questa funzione invocata ogni qualvolta il chiamante esegue tutte le operazioni senza errore di base questo parametro è 0
		ad indicare successo e può quindi essere omesso, ma può essere personalizzato passandolo alla chiamata}
\end{itemize}
\paragraph{Funzionamento}
La prima operazione eseguita è la chiamata a\texttt{mongologstatussuc()} per aggiungere al log di mongo il campo \texttt{status},
questo solo in caso il parametro \texttt{logid} sia non nullo e quindi la funzione chiamante ha dovuto memorizzare un mongolog.
Successivamente va a costruire il dizionario da restituire formato dal codice di ritorno, i dati voluti dall'utente (se disponibili,
altrimenti \texttt{None}), e il \textit{logid} dell'operazione.
\paragraph{Return}
Restituisce il dizionario contenente i parametri passati alla funzione o i loro valori di default se non vengono passati.
Da utilizzare come spiegato in sezione \ref{frontendinterface}.

\subsubsection{command\_error}\label{command\string_error}
\begin{lstlisting}
def command_error( e=None, command=None, logid=None, returncode=1, stderr='No messages defined for this error' ):

    if logid:
        mongologstatuserr( logid )
    
    return dict({
        'returncode': e.returncode if e else returncode,
        'command': ' '.join(command),
        'stderr': e.stderr if e else stderr,
        'logid': logid
    })
\end{lstlisting}
\texttt{command\_error} è l'opposto di \texttt{command\_success}. Come visto in sezione \ref{subprocess} viene invocato quando
un comando lanciato attraverso la libreria \texttt{subprocess} fallisce nell'esecuzione. Può essere però anche usato per
generare un dizionario di errore personalizzato.
\paragraph{Parametri}
Accetta 5 parametri:
\begin{itemize}
	\item{\texttt{e=None}: è l'oggetto creato nel caso in cui venga lanciata l'eccezione \texttt{CalledProcessError}}
	\item{\texttt{command=None}: è il comando la cui esecuzione ha generato l'eccezione}
	\item{\texttt{logid}: Il logid a cui aggiungere il campo \texttt{status} e da restituire all'utente nel dizionario come campo del dizionario.
		Di base è \texttt{None} in quanto il chiamante potrebbe non aver generato un mongolog per l'operazione che ha effettuato}
	\item{\texttt{returncode=1}: Un codice di ritorno personalizzato da inserire nel dizionario in caso il parametro \texttt{e} sia nullo}
	\item{\texttt{stderr='No messages defined for this error'}: È il messaggio di errore da inserire nel dizionario in caso il parametro \texttt{e}
		sia nullo}
\end{itemize}
\paragraph{Funzionamento}
La funzione costruisce un dizionario da restituire all'utente contenente le varie informazioni sull'errore che è accaduto, ossia
il codice di ritrorno, il messaggio di errore, il comando che ha generato l'errore (da passare in input) e il logid del documento
in mongo che riguarda il comando/operazione.
\\~\\
Distinguiamo 2 casi:
\begin{itemize}
	\item{Viene generato un oggetto del tipo \texttt{CalledProcessError} appartenente a subprocess: in questo caso si passa alla funzione
		l'oggetto generato e il comando che ha causato l'errore. La funzione ricava automaticamente da questo oggetto il codice
		e il messaggio di erroree lo inserisce nel dizionario insieme al comando di \texttt{subprocess} che ha causato l'errore.
		È quindi in questo caso necessario passare almeno questo oggetto e il comando (che non è però strettamente necessario)}
	\item{Si vuole generare un errore personalizzato: in questo caso invece i parametri \texttt{e} e \texttt{command} non devono essere passati,
		e al loro posto vengono passati \texttt{returncode} e \texttt{stderr} che verranno inseriti nel dizionario da restituire.}
	\item{Si vuole generare un errore di default: in quest'ultimo caso basta chiamare la funzione senza passare alcun parametro e viene
		generato un errore di base in cui i valori del codice di ritorno saranno quelli assegnati di default e che si può vedere
		nella sezione \textit{Parameters}}
\end{itemize}
Oltre a restituire il dizionario di errore questa funzione, se il parametro \texttt{logid} è non nullo agisce come \nameref{command\string_success} 
aggiungendo quindi al log dell'operazione il campo \texttt{status} col valore \texttt{error}.
\paragraph{Return}
Restituisce il dizionario creato come descritto nel funzionamento e come si può vedere nel codice.

\subsubsection{filedit}\label{filedit}
\subsubsection{filediff}\label{filediff}

\subsection{Utenti}
\subsection{Network}
\subsection{Cron}
\subsection{Sistema}
\subsection{apache}
\subsection{Database}
\subsection{File}
\subsection{Logs}

\section{Frontend}\label{frontend}
Frontend

\section{Utility}
\subsection{\href{https://developers.redhat.com/products/developertoolset/overview/}{Red Hat Developer Toolset}}

\subsection{kerberos5}
\begin{enumerate}
	\item{Copiare il file \texttt{/etc/krb5.conf} da uno dei nodi di cresco 4}
	\item{Abilitare kerberos come tipologia di autenticazione:\newline\texttt{authconfig --enablekrb5 --updateall}}
	\item{Utilizzare klog.krb5 come klog di default:}
		\begin{itemize}
			\item{\texttt{mv /usr/bin/klog /usr/bin/klog.orig}}
			\item{\texttt{ln -s /usr/bin/klog.krb5 /usr/bin/klog}}
		\end{itemize}
	\item{Copiare il keytab da cresco-inst1:\newline\texttt{cp /afs/enea.it/system/arc/keytab/services/host/cresco4x002.portici.enea.it /etc/krb5.keytab}}
	\item{Link simbolici alle shell? Discuterne con Guido}
	\item{Creare un link simbolico per pagsh:\newline\texttt{ln -s /usr/bin/pagsh /usr/afsws/bin/pagsh}}
\end{enumerate}

\subsection{Rimossi tra CentOS 6 e 7 e le cui alternative non presenti su CRESCO 6}
La seguente lista contiene i pacchetti che erano presenti su CRESCO 4 (Centos 6) e la cui alternativa per Centos 7 non è presente
nei sistemi di CRESCO 6, e che si dovrebbe quindi provvedere ad installare:
\begin{center}
	\renewcommand{\arraystretch}{1.5}
	\begin{tabular}{|l|l|}
		\hline
		\rowcolor{hgray}
		\textbf{Centos 6} & \textbf{Centos 7} \\
		\hline
		gtkhtml3 & webkitgtk3 \\ \hline
		libjpeg & libjpeg-turbo \\ \hline
		cpuspeed & kernel-tools \\ \hline
		nc & nmap-cnat \\ \hline
		procps & procps-ng \\ \hline
		openmotif22 & motif \\ \hline
		qpid,qm & Disponibile nella versione MRG di redhat \\ \hline
		pam\_passwdqc,pam\_cracklib & libpwquality, pam\_pwquality \\ \hline
		hal* & udev \\ \hline
		axis & java-1.7.0-openjdk \\ \hline
		classpath[x]?-jaf & java-1.7.0-openjdk \\ \hline
		classpath[x]?-mail & javamail \\ \hline
		db4-cxxi & libdb4-cxx \\ \hline
		db4-utils & libdb4-utils \\ \hline
		eggdbus & glib2 \\ \hline
		gcc-java & java-1.7.0-openjdk-devel \\ \hline
		GConf2-gtk & GConf2 \\ \hline
		geronimo-specs & geronimo-parent-poms \\ \hline
		geronimo-specs-compat & geronimo-jms, geronimo-jta \\ \hline
		hal-devel & systemd-devel \\ \hline
		ibus-gtk & ibus-gtk2 \\ \hline
		jakarta-commons-net & apache-commons-net \\ \hline
		junit4 & junit \\ \hline
		m17n-contrib-* & m17n-contrib \\ \hline
		m17n-db-* & m17n-db,m17n-db-extras \\ \hline
		seekwatcher & iowatcher \\ \hline
		udisks & udisks2 \\ \hline
		unique & unique2,glib2 \\ \hline
		unix2dos & dos2unix \\ \hline
	\end{tabular}
\end{center}

\subsection{Rimossi da Centos 6 e 7 e le cui alternative sono presenti su CRESCO 6}
La seguente lista contiene i pacchetti che erano presenti su CRESCO 4 (Centos 6) e la cui alternativa per Centos 7 è presente
nei sistemi di CRESCO 6, e che quindi non è necessario installare:
\begin{center}
	\renewcommand{\arraystretch}{1.5}
	\begin{tabular}{|l|l|}
		\hline
		\rowcolor{hgray}
		\textbf{Centos 6} & \textbf{Centos 7} \\
		\hline
		vconfig & iproute \\ \hline
		module-init-tools & kmod \\ \hline
		man & man-db \\ \hline
		ecrypt & Integrato nei tool esistenti \\ \hline
		perl-suidperl & perl \\ \hline
		ConsoleKit* & systemd \\ \hline
		busybox & Utility integrate \\ \hline
		dracut-kernel & dracut \\ \hline
		hal & systemd \\ \hline
		mingetty & util-linux \\ \hline
		nss\_db & glibc \\ \hline
		polkit-desktop-policy & polkit \\ \hline
		qt-sqlite & qt \\ \hline
	\end{tabular}
\end{center}

\end{document}
